\documentclass[aps,prl,reprint,superscriptaddress,floatfix]{revtex4-1}

%--- PACKAGES ---%
% \usepackage[T1]{fontenc}
\usepackage{graphicx}
\usepackage{color}
\usepackage{amsmath,amssymb}
\usepackage{bm}
\usepackage{upgreek}
\usepackage{xspace}
\usepackage{units}
\usepackage[colorlinks,urlcolor=blue,citecolor=blue,linkcolor=blue]{hyperref}
\graphicspath{{figures/}}

%--- TEXT SHORTCUTS ---%
\newcommand{\etal}{et~al.\xspace}
\newcommand{\Rb}{$^{87}$Rb\xspace}

%--- TEXT OPERATORS ---%
\newcommand{\reffig}[1]{\mbox{Fig.~\ref{#1}}}
\newcommand{\refeq}[1]{\mbox{Eq.~(\ref{#1})}}

%--- MATH OPERATORS ---%
\newcommand{\upd}{\text{d}}
\newcommand{\totalD}[2]{\frac{\upd #1}{\upd #2}}
\newcommand{\partialD}[2]{\frac{\partial #1}{\partial #2}}
\newcommand{\vecop}[1]{\hat{\bum{#1}}\xspace}
\newcommand{\expect}[1]{\langle #1 \rangle}
\newcommand{\abs}[1]{\vert #1 \vert \xspace}
\newcommand{\bra}[1]{\langle #1 \vert \xspace}
\newcommand{\ket}[1]{\vert #1 \rangle \xspace}
\newcommand{\braket}[2]{\langle #1 \vert #2 \rangle \xspace}
\newcommand{\vect}[1]{\boldsymbol{#1}\xspace}
\newcommand{\uvect}[1]{\hat{\boldsymbol{#1}}\xspace}
\newcommand{\epvec}{\hat{\boldsymbol{\epsilon}}\xspace}
\newcommand{\jvect}{\boldsymbol{\tilde{E}}\xspace}

% Underscores in text (from http://tex.stackexchange.com/a/38720)
\catcode`_=12
\begingroup\lccode`~=`_\lowercase{\endgroup\let~\sb}

\begin{document}

\title{Continuously observing the spectrum of a dynamically decoupled spin-1 quantum gas}

\author{R.\,P.~Anderson}
\author{M.\,J.~Kewming }
\author{L.\,D.~Turner}
\affiliation{School of Physics \& Astronomy, Monash University, Victoria 3800, Australia.}

\date{\today}

\begin{abstract}
Quantum states and spectra can be made sensitive to a particular measurand whilst simultaneously impervious to parasitic fluctuations of an environment.
Here we use an atom-light interface with minimal backaction to probe the spectrum of a radiofrequency-dressed spin-1 quantum gas continuously and in-situ.
The dressing amplitude sets the radiofrequency band in which oscillating magnetic fields manifest a linear measurand, and we probe the energy spectrum while the system evolves unitarily.
By varying a symmetry-breaking parameter of the Hamiltonian, we find a regime in which two of the dressed states are maximally insensitive (up to fourth-order) in magnetic field fluctuations that are slow compared to the dressed-state splittings.
Moreover, we demonstrate the predictive power of our continuous probe to optimize the dynamical decoupling and tune the measurement band.
This robust system shares the useful hallmarks of quantum metrology platforms; the states are thus termed ``synthetic clock'' states in a co-published result by Lundblad~\textit{et al.} (Phys. Rev. Lett. \textbf{118}, 2xxxxx (2017)) and are candidates for band-tunable magnetometry and color charge analogues in quantum gases. 
\end{abstract}

\maketitle

\section{Introduction}
\label{sec:introduction}
From Hahn echoes to Uhrig dynamical decoupling to the pulse sequence \textit{du jour}, abrupt, discrete rotations have been used to protect spin superpositions from inhomogeneities and parasitic fluctuations, prolonging quantum coherence and circumventing deleterious energy shifts \note{nmr,qi,ions}.
A complementary strategy is to replace the pulse train with an uninterrupted coupling of `bare' spin states, thus admitting new `dressed' spin eigenstates, with a modified quantization direction, spectrum, and coupling, which too are protected from unwanted artifacts of their environment.
This \textit{continuous} dynamical decoupling (CoDD) has proven useful across multiple platforms including nitrogen-vacancy centers and superconducting qubits, and forms the basis for creating protected qubit subspaces~\cite{aharon_general_2013}.
Weak continuous measurement is the natural instrument for appraising and refining dynamical decoupling in real time; to probe stochastic evolution, improve metrological bandwidth, or realize quantum feedback schemes.
Here we use a minimally perturbative \note{dispersive} optical probe of a `bare' spin component $\hat{F}_x$ to measure the spectrum of a continuously decoupled spin-1 ensemble in real-time.
The rich time-frequency domain data reveal not only multiple dressed-state splittings and their relative immunity to noise, but also dressed state coherences and coupling strengths.
This potent ability to estimate the eigenspectrum of a multi-level dressed system reveals surprising features absent in spin-1/2 or composite spin-1/2 qubit systems; we identify a regime in which a subspace of the dressed system is maximally decoupled from noise $\propto \hat{F}_z$.
This protected subspace is spanned by two of the dressed states, termed `synthetic clock states' in a co-publication by Lundblad~\textit{et al.} (Phys. Rev. Lett. \textbf{118}, 2xxxxx (2017)).
% the spin-1 asymmetry arising from $\hat{F}_z^2$ relative to the 

\subsection{Literature review notes}
\begin{itemize}
    \item Minimally insensitive states in other systems, e.g. $\ket{F=1,m=-1} \leftrightarrow \ket{F=2,m=+1}$ at $B=\unit[3.23]{G}$~\cite{matthews_dynamical_1998}, clocks.
    \item Wide utility of these states for clocks, magnetometers (including microwave, e.g. Treutlein), quantum information, and quantum emulation, e.g. elusive many-body spin-singlet state which requires unforgiving field stability.
    
    \item In electronically/magnetically sensitive spin experiments, magnetic field noise manifest as small perturbations $\Delta \hat{F}_{z}$ which shift the energy eigenvalues.
    Continuous measurement of and $F=1$ system using the Faraday effect necessitates the state be magnetically sensitive $\langle \hat{F} \rangle \neq 0$ but dark states, protected from magnetic field fluctuations only exist in the nullspace of $\hat{F}_{z}$ i.e. $\hat{F}_{z}\ket{\psi} = \hat{0}$ \cite{aharon_general_2013}. 
    Therefore, no state exists which is both insensitive to magnetic fields fluctuations and capable of generating a continuous Faraday signal, the two are not mutually exclusive.

    It has been demonstrated that Continuous dynamical decoupling (CoDD)~\cite{facchi_unification_2004,*fanchini_continuously_2007} occurs when the spin undergoes rotations in a plane perpendicular induce perturbation, i.e $\hat{F}_{x}$ or $\hat{F}_{x}$ most readily achieved through Rabi coupling where $\Omega \gg \Delta B$.
    Decoupling can be used to increase the coherence time of single qubits~\cite{golter_protecting_2014} which can be extended even further by doubly dressing the system and decoupling from noise induced by the first coupling field~\cite{cai_robust_2012} (concatenated continous decoupling, or CCD).
    It has also been shown theoretically that CoDD is superior than pulsed sequence dynamical decoupling for single solid-state qubits for magnetometery~\cite{hirose_continuous_2012}.
    \note{There is a wealth of literature on this topic in NV. Should trace down the biggest results in this field and cite them~\cite{cai_long-lived_2012}.}
\begin{figure}
    \centering
    \includegraphics[width=\columnwidth]{figure_1.pdf}
    \caption{
    \label{fig:eigensystem_schematic}
        (Color online)
        Energy spectrum and splittings of a radiofrequency coupled spin-1 for various $q(B)\in[0,\Omega]$.
        The transparency of each curve is proportional to the distance of the quadratic shift $q$ from $\qmagic\approx 0.348\Omega$.
        (Left) Energies $\omega_n$ of dressed states $\ket{n}=\ket{1}$, (red) $\ket{2}$ (blue), and $\ket{3}$ (green) normalized to the rf-coupling strength (Rabi frequency) $\Omega$ as a function of detuning $\Delta(B)=\omega_{\text{rf}}-\omega_L(B)$,
        Dashed lines indicate the energies of uncoupled states ($\Omega=0$) in a frame rotating at $\omega_{\text{rf}}$.
        (Right) Splittings $\omega_{ij}$ of dressed states $\ket{i}$ and $\ket{j}$ as a function of detuning.
        When $q=\qmagic$ (bold curves), energies $\omega_1$ and $\omega_2$ share the same curvature, and their difference $\omega_{12}$ (right, purple) is minimally sensitive to detuning and thus magnetic field variations. 
    }
\end{figure}

    \item Motivate continuous measurement, especially in context of measurement bandwidth; it doesn't make sense to measure something in kHz--MHz band using a shot-based ($\unit[0.1]{Hz}$ or less) readout. Why? Can't react, can't feedback, can't always assume periodicity/repeatability.
    \item Whilst the paper is not focused on introducing spectrograms, we can say they provide a new mechanism for appraising spectrum of a quantum system in real-time. 
    \item From a magnetometry perspective, breaking rotational symmetry is bad because you want there to be no anisotropy to the sensitivity. How does this relate to this work?
    \item \note{Not sure about this but many NV center guys study Luminescence as a function of time. Could they make spectrograms with there data? See data from~\cite{cai_robust_2012}.}
    \item (How) The $\hat{F}_z^2$ interaction has been controlled using static magnetic fields, microwave ac Stark shifts and tensor-light shifts of off-resonant magnetic and electric dipole transitions, respectively. In collective psuedo-spins, it can arise from nonlinear collisional interactions~\cite{riedel_atom-chip-based_2010,*gross_nonlinear_2010}.
    \item (What) This nonlinear term has been used to traverse the magnetic phase space of a spinor quantum gas, drive quantum quenches of same, initiate spin dynamics in optical lattices~\cite{gerbier_resonant_2006}, enact the canonical spin-squeezing of Kitagawa and Ueda~\cite{kitagawa_squeezed_1993}: one-axis twisting (shear of coherent spin state uncertainty regon) that has squeezed atomic spins in cavitities~\cite{leroux_implementation_2010}, Bose-Einstein condensates~\cite{riedel_atom-chip-based_2010,*gross_nonlinear_2010}, and superconducting qubits.
\end{itemize}
\textit{Continuous measurement using the linear Faraday effect:}
\begin{itemize}
    \item Paramagnetic Faraday effect has been used to continuously measure spin-mixing dynamics of a polar spinor condensate~\cite{liu_quantum_2009}.
    \item Quantum state tomography (QST)~\cite{smith_continuous_2004,*smith_efficient_2006} using continuous weak measurement and dynamical control.
    Ref.~\cite[2004]{smith_continuous_2004} foreshadows ``real-time estimation of the spin density matrix'' via weak measurement.
    \item Real-time nonperturbing polarization probe (tensor-based at $\Delta \sim \Delta_{HF}$, not Faraday) in~Ref.~\cite{chaudhury_continuous_2006} measured hyperfine Rabi oscillations of the collective clock-transition pseudospin.
    The birefringence is modulated near baseband, i.e. polarization rotation oscillates at the Rabi frequency.
    This paper highlights the fact that ``atoms in $m_F=0$ clock states cannot contribute to Faraday rotation, and therefore in the limit $\Delta \gg \Delta_{HF}$ the probe polarization does not couple to the clock pseudospin''.
    In contrast, the synthetic clock states presented here can yield Faraday rotation, including in the $\Delta \gg \Delta_{HF}$ limit.
    Moreover, the probe-induced differential light shift changed the resonance of the clock-transition (detected as a modified total Rabi frequency).
    Also from this paper:
    \begin{quoting}
        \dots if a measurement can resolve the quantum fluctuations associated with a collective observable, then backaction will be induced on the collective state and the observable can be squeezed.
    \end{quoting}
    \item Faraday QND measurement of collective spin projection of an atomic beam as a way of preparing squeezed spin states~\cite{kuzmich_quantum_1999,*kuzmich_generation_2000}.
    This is an odd regime where the background field is modulated faster than the peak Larmor frequency, but it is a rare example of the polarimetry signal being fed to a spectrum analyzer/lock-in detector.
\end{itemize}
\textit{Other exotica:}
\begin{itemize}
    \item Spin-orbit coupled spin-1 Bose gases exhibit e.g. tricriticality in $(\Omega, q)$ phase diagrams~\cite{martone_tricriticalities_2016}, where the optimal decoupling $q = \qRmagic \Omega$ is a line traversing multiple phases, in the vicinity of a tricritical point of polar-striped and plane-wave phases. Although here the wavevector of the coupling is zero, the essential results may carry over, and the Faraday probe -- used to detect magnetization -- could constitute one of the Raman beams used to generate the spin-orbit coupling.
    \item Lipkin-Meshkov-Glick Hamiltonian~\cite{lipkin_validity_1965} (see Refs.~[16] of~\cite{muessel_twist-and-turn_2015}). 
\end{itemize}

\section{Background + random}
\label{sec:background}
\textit{Structure of background theory:}
\begin{itemize}
    \item Introduce Hamiltonian for CoDD (clock states have been introduced so splitting will come naturally).
    \item Show eigenenergies of for varying $q$ and seperations.
    \item Point out in figure that curvature of $\omega_{12}$ starts positive but becomes negative.
    By mean value theorem, it must become vanishing.
    \item Use second order perturbation theory to derive $\qmagic$.
\end{itemize}
Hamiltonians (quasi-static field along $z$, coupling field along $x$:
    \begin{align*}
        \hat{H}_{\text{lab}} &= -\omega_L \hat{F}_z + q \hat{F}_z^2 + 2\Omega \cos (\omega_{\text{rf}} t) \hat{F}_x \\
        \Rightarrow \hat{H}_{\text{rwa}} &= \Delta \hat{F}_z + q \hat{F}_z^2 + \Omega \hat{F}_x \, \text{, where} \\
        \omega_L(B) &\equiv (E_{m=-1} - E_{m=+1})/2\hbar \, \text{, and} \\
        q(B) &\equiv (E_{m=+1} + E_{m=-1} - 2 E_{m=0})/2\hbar
    \end{align*}
    are the Larmor frequency and quadratic shift, respectively, which can be gleaned from the Breit-Rabi equation.
    The rf Rabi frequency $\Omega = \gamma B_{\text{rf}}/2$ and detuning $\Delta(B) = \omega_{\text{rf}} - \omega_L(B)$.
\begin{figure}
    \centering
    \includegraphics[width=\columnwidth]{figure_2.pdf}
    \caption{
    \label{fig:static_coupling}
    Continuous measurement of the dressed energy spectra for $q_R = 0.402(3)$, $f_{\text{rf}}=\unit[3.521]{MHz}$ and $B_0=\unit[5.013]{G}$ (left) and the log of the fitted moving average signal strength in normalized units (right). 
    The shape of the energy spectra in the Fourier domain are Gaussian, but magnetic field fluctuations skew the time average shape.  
    Skewed Gaussian curves are fitted to each peak resulting in a significant skew parameter of $\gamma = 2.4(1)$ for the $\omega_{23}$ splitting width $\sigma_{23} = \unit[243(2)]{Hz}$. 
    The fit of the transition most decoupled from magnetic field fluctuations $\omega_{12}$ has a vanishing skew and is well approximated by a Gaussian of width $\sigma_{12} = \unit[109(1)]{Hz}$.
    }
\end{figure}

The control field $\vect{B}(t \geq 0) = -B_{\text{rf}} \cos (\omega_{\text{rf}} t) \vect{e}_x + B_z(t) \vect{e}_z$, where $B_z(t) = B_0 + \alpha t + B_{\text{ac}}(t)$ is the sum of a linear ramp and parasitic power-line magnetic noise.

\begin{figure}
    \centering
    \includegraphics[width=\columnwidth]{figure_3.pdf}
    \caption{
    \label{fig:acquisition_pipeline}
        (Color online)
        Acquisition and analysis pipeline of the continuous spectrum observation for $q_R = 0.402(3)$ ($f_{\text{rf}}=\unit[3.521]{MHz}$, $B_0=\unit[5.013]{G}$).
        The underlay of (A) and (B) are spectrograms of raw acquisitions from the polarimeter measuring Faraday rotation of the probe beam.
        (A) Time-resolved magnetometry is used to calibrate the instantaneous magnetic field $B(t) = B_0 + \Delta B(t)$ over the interrogation interval, in which the field [detuning] varies over a range $\sim B_{\text{rf}}$ [$2\Omega$] and the spinor gas is left to Larmor precess in the absence of a dressing field ($\Omega = 0$).
        We fit sinc peaks in the frequency domain for each spectrogram window (time domain) to determine the instantaneous Larmor frequency $f_L(t)$ and quadratic shift $q(t)$.
       (B) The field is swept over the same range but the radiofrequency dressing is applied ($\Omega > 0$).
       Three sidebands appear above (shown) and below the carrier at $f_{\text{rf}}$ (dashed, orange), revealing the dressed state splittings $\omega_{ij}$.
       (C) A parametric plot of $f_{12}(t)$ and $f_{23}(t)$ versus $\Delta B(t)$ by combining analysis of (A) and (B).
       Solid curves in (B) and (C) are theoretical splittings from an eigenspectrum calculation, provided only $f_{\text{rf}}$, $B(t)$, and $\Omega$, i.e. no free parameters.
       The synthetic clock transition $f_{12}$ varies by $\unit[39]{Hz}$ for $0 \leq \Delta B \leq B_{\text{rf}}/4 = \unit[3.2]{mG}$, equivalent to $0 \leq \abs{\Delta/\Omega} \leq 0.5$ (C, inset).
    }
\end{figure}

\begin{itemize}
    \item At low magnetic field strengths, $\omega_L \propto B$ and $q \propto B^2$, and for our parameters we are justified in taking $\omega_L = \gamma B$ and $q = q_Z B^2$, where $\gamma = 2\pi \times \unit[702.379]{kHz/G}$ is the gyromagnetic ratio for \Rb $F=1$ and $q_Z = 2\pi \times \unit[71.89]{Hz/G^2}$.
    \item For most of the analysis presented here (with $\omega_L$ and $q$ defined as above) these proportionalities need not be met, or the results, e.g. value of $\qmagic$ require a small correction. 
    \item For $q=0$, $\hat{H}_{\text{lab}}$ and $\hat{H}_{\text{rwa}} \propto \vect{B} \cdot \hat{\vect{F}}$ and are thus a generator of rotations, but $q \hat{F}_z^2 \neq 0$ breaks the $\text{SU}(2)$ symmetry and $\expect{\vect{\hat{F}}}^2$ is not preserved.
    \item This broken symmetry lifts the degeneracy of the dressed state splittings making them distinguishable in our spectrogram measurements.
    \item We vary the magnetic field to affect a change in the detuning of $\Delta \in [0, 2\Omega]$, the domain of \reffig{fig:eigensystem_schematic}(B).
    \item Variations in $B \mapsto B_0 + \Delta B$ of order $B_{\text{rf}} = 2\Omega/\gamma$ affect the detuning linearly, \textit{viz.} $\Delta = - \gamma \Delta B$ for $\omega_L(B_0) = \omega_{\text{rf}}$ (i.e. rf is resonant when $B=B_0$), and do not affect $q$ appreciably for sufficiently small field strengths.
        \item Indeed, our data corroborate this since we directly measure $q(B(t))$ during the field sweep and find that $\sigma(q)/(2\pi) = \unit[11.7]{Hz}$ on average (alternatively, inferring $q$ from $\omega_L$ via the Breit-Rabi equation gives $\sigma(q)/(2\pi) = \unit[1.2]{Hz}$).
    \item Thus the horizontal axis $\Delta/\Omega$ in \reffig{fig:eigensystem_schematic} is a proxy for $\Delta B$, and $\omega_{12}$ at $q=\qmagic$ has leading-order quartic sensitivity to $\Delta B$.
        \item At high fields ($B\approx \unit[30]{G}$) this approximation is no longer valid; $q \approx \qmagic$ varies appreciably across $\Delta B \in [0, B_{\text{rf}}]$ and $\omega_{12}$ has weak linear dependence on $\Delta B$~\cite{lundblad_synthetic_2017}.
    \item \textit{On varying $\Omega$ or $q$ to change $q_R=q/\Omega$:}
    For a given static magnetic field, $q_R$ can be modified via the Rabi frequency.
    However, this is not what is represented in \reffig{fig:eigensystem_schematic}, as the normalization of the horizontal and vertical axes would vary for each $q_R$.
    Importantly, the insensitivity of $\omega_{12}$ to detuning only gets better for increasing $\Omega$ in absolute terms; if the rf amplitude is unlimited, use it.
    However, doing so also modifies the absolute dressed state splittings on resonance, and thus the bandwidth of the dressed spin-1 as an ac magnetometer.
    The take home message is then: use as high an rf amplitude as you can afford (or want to tune the ac-band to), and then modify $q_R$ via $q$ to realize the synthetic clock states.
    \item \textit{Transitions between dressed states:} $\ket{1} \leftrightarrow \ket{2}$ and $\ket{2} \leftrightarrow \ket{3}$ driven by fields oscillating along $y$ or $z$ near frequencies $\omega_D \mp q_D$, respectively.
    Alternatively, $\ket{1} \leftrightarrow \ket{3}$ driven by fields oscillating along $x$ near frequency $2\omega_D$.
    This is very different to the fully polarized bare states $\ket{m=\pm 1}$, which are coupled by a single-photon transition as this would conserve neither photon number nor angular momentum.
    There is no such restriction on the dressed states however as they are neither eigenstates of $\hat{F}_z$ nor photon number. 
\end{itemize}
\textit{Lab frame eigenstates:}
\begin{itemize}
    \item Mean splitting $\omega_L$; quadratic shift $q$.
    \item Pairwise coupling between $\ket{m=-1} \leftrightarrow \ket{m=0}$ and $\ket{m=0} \leftrightarrow \ket{m=+1}$ via $\hat{F}_x$ and/or $\hat{F}_y$, i.e. affected by fields transverse to the static field oscillating near $\omega_L$.
    \item The $F=1$ spin operators in the undressed basis do not span all of $\text{SU}(3)$, i.e. $\sigma_x = (\lambda_1 + \lambda_6)/\sqrt{2}$, $\sigma_y = (\lambda_2 + \lambda_7)/\sqrt{2}$, and $\sigma_z = (\lambda_3 + \sqrt{3} \lambda_8)/2$ ($\lambda_4$ and $\lambda_5$ which include off diagonal terms in rows/columns 1 and 3 are absent) where $\lambda_i, i = 1,\dots,8$ are the Gell-Mann matrices.
\end{itemize}
\textit{Dressed states on resonance:}
\begin{itemize}
    \item Dressed Larmor frequency:
    \begin{align*}
        \omega_D \equiv (\omega_3 - \omega_2)_{\Delta=0}/2 &= (\omega_{12} + \omega_{23})_{\Delta=0}/2 \\ &= \sqrt{\Omega^2 + q_D^2}\, .
    \end{align*}
    \item Dressed quadratic shift:
    \begin{align*}
        q_D &\equiv (\omega_3 + \omega_1 -2\omega_2)_{\Delta=0}/2 \\
            &= (\omega_{23}-\omega{12})_{\Delta=0}/2\\ 
            &= -q/2 \, .
     \end{align*}
    \item Thus $\Omega = \sqrt{\omega_{12} \omega_{23}}_{\Delta=0}$ and $q_D = (\omega_{23} - \omega_{12})_{\Delta=0}/2$, both of which can be attained from the dressed sideband splittings on resonance.
    Such high-bandwidth measurement of $\Omega$ (magnetic field oscillating along $x$ with amplitude $B_{\text{rf}}$ and frequency $\omega_L$) allows (in principle) closed-loop control of $\Omega$ using the atoms.
    \item For $q=0$ (low-field limit), the dressed states at $\Delta=0$ are eigenstates of $\hat{F}_x$, \textit{viz.} $\ket{m_x=-1,0,+1} \equiv \ket{1}$, $\ket{2}$, and $\ket{3}$ respectively.
    (i) the spectrum has vanishing linear sensitivity to magnetic fields, with the leading quadratic sensitivity (as in spin-$1/2$) of the degenerate $\ket{1} \leftrightarrow \ket{2}$ and $\ket{2} \leftrightarrow \ket{3}$ transitions, and (ii) fields along $y$ or $z$ oscillating near the Rabi frequency $\Omega$ drive transitions between different $\ket{m_x}$ states. \note{Cite other dressed-ception papers on both of these.}
    \item The $F=1$ spin operators in the resonantly dressed basis span more of $\text{SU}(3)$, as $[\hat{F}_x]_D = (q_D/\omega_D) Q_{x^2-y^2} - (\Omega / \omega_D) \sigma_z = (q_D/\omega_D) \lambda_4 - \Omega / \omega_D ( \lambda_3 +\sqrt{3} \lambda_8) / 2$, $[\hat{F}_y]_D$ is a sum of $\lambda_1, \lambda_2$, and $\lambda_7$, and  $[\hat{F}_z]_D$ is a sum of $\lambda_1$ and $\lambda_6$.
    The presence of $Q_{x^2-y^2} = \lambda_4$ for $q \neq 0$ signifies the coupling of dressed states $\ket{1}$ and $\ket{3}$, i.e. a non-zero $\bra{1} \hat{F}_x \ket{3}$.
    \note{The spin operators have no projection onto $\lambda_5$, even for an rf field oscillating along $y$.}
    \item Curvature of the dressed-state energies can be evaluated using perturbation theory;
    \[
    \partialD{^2\omega_n}{\Delta^2} = \sum_{k \neq n} \frac{\abs{\bra{k} \hat{F}_z \ket{n}}^2}{\omega_n - \omega_k} \, .
    \]
    \item Thus the curvature of the dressed-state splittings can be found. In particular, the dimensionless curvature of $\omega_{12}$ is (presuming $\abs{\partial q / \partial \Delta} \ll 1$) 
    \begin{align*}
    \partialD{^2(\omega_{12}/\Omega)}{(\Delta/\Omega)^2} &= \Omega \partialD{^2\omega_{12}}{\Delta^2} \\ &= -\frac{3 q_R \sqrt{4 + q_R^2} - q_R^2 - 2}{2 \sqrt{4 + q_R^2}} \, .
    \end{align*}
    This vanishes when $q = \qRmagic$, given by
    \[
    \qRmagic = \sqrt{(3\sqrt{2} - 4)/2} \approx 0.348 \, .
    \]
    For $q_R = 0$, we recover the spin-1/2 result, $\Omega\, \partial^2\omega_{12}/\partial \Delta^2 = 1$.
    \item Similarly, perturbation theory can be used to show that the third-order derivatives of $\omega_n$ with respect to detuning all vanish when $\abs{\partial q / \partial \Delta} \approx \abs{\gamma^{-1} \partial q / \partial B} \ll 1$, and thus the leading sensitivity to detuning (and thus $B$) is fourth-order. This validates the choice of our phenomenological even-polynomial model for fitting to $(\Delta B(t), \omega_{12}(t))$ data extracted from Faraday spectrograms.
    \item The above can be quantified by noting that $\gamma^{-1} \partial q / \partial B = 2 B q_Z / \gamma \lesssim 10^{-3}$ for $B \lesssim \unit[5]{G}$.
    \item Near $q = \qmagic$, the ratio of the Rabi frequency to the Larmor frequency is approximately:
    \begin{align*}
        \frac{\Omega}{\omega_L} &= \frac{B_\text{rf}}{B_0} \\
                                &\approx \frac{q_z B_0}{\sqrt{2} \gamma \, \qRmagic} \\
                                &= 2.1\times10^{-4} B_0 \, ,
    \end{align*}
    with $B_0$ is in Gauss.
    Thus for $B_0 \lesssim \unit[5]{G}$, $\Omega/\omega_L \lesssim 10^{-3}$ and the rotating-wave approximation is justified.
    \item For $0 \leq \Delta B \leq B_{\text{rf}}/4 = \unit[3.2]{mG}$ ($0 \leq \abs{\Delta/\Omega} \leq 0.5$) we observe a variation in the splitting $f_{12}$ of $\unit[39]{Hz}$ for the data in \reffig{fig:acquisition_pipeline}, compared to the theoretical estimate of $\unit[26]{Hz}$.
    These correspond to a normalized variation in $\omega_{12}/\Omega$ of $8.6\times10^{-3}$ and $5.8\times10^{-3}$, respectively.
    By comparison, the normalized variation at $q_R=0$ is $(\sqrt{5}-2)/2 \approx 0.118$; 14 [20] times higher than the observed [predicted] variation.
    Alternatively, the normalized variation of the $\ket{m=\pm1} \leftrightarrow \ket{0}$ transitions at $q=0$ is $0.5$; 58 [86] times higher than the observed [predicted] variation in the synthetic clock transition frequency.
    \note{Both of these comparisons depend a lot on the range of $\Delta B$! They are far more impressive (theoretically) for smaller ranges.}
\end{itemize}
\textit{Analytic Faraday signal on resonance:}
\begin{table}[]
\centering
\caption{Upper sidebands of the carrier (at $\omega_{\text{rf}}$) of the Faraday rotation signal $\propto \expect{\hat{F}_x}$ of a state $\ket{\psi(t=0)}=\ket{m=1}$ driven on resonance ($\Delta = 0$) in the laboratory frame. Frequency and phase are reported relative to the carrier, along with the transition that each sideband corresponds to. For each upper sideband, there is a lower sideband of the same amplitude, relative frequency and opposite \note{$\pi$} relative phase.}
\label{tab:sidebands}
\begin{tabular}{cccc}
frequency & amplitude & transition \\ \hhline{====}
 $0$ & $\frac{q_D \Omega }{2 \omega_D^2}$ & -- \\
 $\omega_D-q_D$ & $\frac{\Omega}{4 \omega_D}$ & $\ket{1} \leftrightarrow \ket{2}$ \\
 $\omega_D+q_D$ & $\frac{\Omega}{4 \omega_D}$ & $\ket{2} \leftrightarrow \ket{3}$\\
 $2 \omega_D$ & $\frac{q_D \Omega}{4 \omega_D^2}$ & $\ket{1} \leftrightarrow \ket{3}$
\end{tabular}
\end{table}
The Faraday signal we detect is proportional to $\expect{\hat{F}_x}$ in the laboratory frame, which is given by $\bra{\psi(t)}\hat{S}^{\dagger} \hat{F}_x \hat{S} \ket{\psi(t)}$ where $\ket{\psi(t)} = \exp(-i \hat{H}_{\text{rwa}} t/\hbar) \ket{\psi(t=0)}$, and $\hat{S} = \exp(-i \omega_{\text{rf}} \hat{F}_z t)$.
For an initially polarized $\ket{\psi(t=0)} = \ket{m=1}$ state driven on resonance, we get
    \begin{align*}
        \expect{\hat{F}_x}_{\text{lab}} &= -\frac{\Omega}{\omega_D} \cos(q_D t) \sin(\omega_D t) \sin(\omega_{\text{rf}} t) - \\ &\frac{q_D \Omega}{\omega_D^2} \sin^2(\omega_D t) \cos(\omega_{\text{rf}} t) \, .
    \end{align*}
The first term has equal-amplitude sidebands at $\pm(\omega_D \pm q_D)$ and the second term has smaller amplitude sidebands at $\pm 2\omega_D$, as summarized in Table~\ref{tab:sidebands}.
The ratio of the $\omega_{13}$ sideband amplitude to the $\omega_{12}$ and $\omega_{23}$) sideband amplitudes is $q_R/\sqrt{4+q_R^2} \approx 0.197$ for $q_R = 0.402$ in \reffig{fig:acquisition_pipeline}.
For the data in \reffig{fig:static_coupling} we measure the ratio of sideband amplitudes to be $0.699$ and $0.210$ of the $\omega_{23}$ and $\omega_{13}$ peaks relative to the $\omega_{12}$ peak respectively.

\textit{Faraday signal for arbitray states:}
Evaluating $\expect{\hat{F}_x}_{\text{lab}}$ in the dressed basis, one can show analytically that the sideband frequencies are the dressed state splittings $\omega_{ij}$ with amplitudes proportional to the relevant dressed-state coherence $\text{Re}\{\rho_{ij}\}$ where $\rho_{ij} = \braket{i}{\psi}\braket{\psi}{j}$.
The carrier amplitude is $(\rho_{33} - \rho_{11}) \Omega / \omega_D$.
% (\abs{\braket{3}{\psi}}^2 - \abs{\braket{1}{\psi}}^2) \Omega / \omega_D =
This explains the relative longevity of the sidebands; the transition most sensitive to detuning is $\omega_{13}$, and thus the coherence $\rho_{13}$ decays the fastest in the presence of detuning noise with a measured $/1e$ signal decay time of $\tau_{s}=44(1)\unit{ms}$ from \reffig{fig:static_coupling}. 
Furthermore, the decay time of the $\omega_{12}$ and $\omega_{23}$ sidebands are $74(1)\unit{ms}$ and $71(2)\unit{ms}$ respectively. 
Constrasting these results to an undressed measurement decay time $\tau_{s}=21(1)$, it is clear that rf dressing shields the quantum state from magnetically induced decoherence.
The constant of proportionality is the coupling strength of the dressed states.
\begin{table*}
\label{tab:sidebands_arb}
\begin{tabular}{c|c|c}
transition & sideband frequency & sideband amplitude \\ \hhline{===}
 -- & $\omega_{\text{rf}}$ & $(\bra{3} \hat{F}_x \ket{3} - \bra{1} \hat{F}_x \ket{1}) (\rho_{33} - \rho_{11})$ \\
 $\ket{1} \leftrightarrow \ket{2}$ & $\omega_{\text{rf}} + \omega_{12}$ & $-2i \bra{1} \hat{F}_y \ket{2} \text{Re}\{\rho_{12}\} = -2 \bra{2} \hat{F}_z \ket{3} \text{Re}\{\rho_{12}\} $ \\
 $\ket{2} \leftrightarrow \ket{3}$ & $\omega_{\text{rf}} + \omega_{23}$ & $2i \bra{2} \hat{F}_y \ket{3} \text{Re}\{\rho_{23}\} = 2 \bra{1} \hat{F}_z \ket{2} \text{Re}\{\rho_{23}\}$ \\
 $\ket{1} \leftrightarrow \ket{3}$ & $\omega_{\text{rf}} + \omega_{13}$ & $2 \bra{1} \hat{F}_x \ket{3} \text{Re}\{\rho_{13}\}$
\end{tabular}
\end{table*}

\textit{Adiabatic considerations:}
The rf-dressing is applied non-adiabitcally, and as a result we project $\ket{m=-1}$ onto the dressed basis at the initial detuning $\Delta(t=0)$ into a state
\begin{equation*}
    \begin{pmatrix}
        \braket{1}{\psi(t=0)} \\
        \braket{2}{\psi(t=0)} \\
        \braket{3}{\psi(t=0)} 
    \end{pmatrix} \approx 
    \begin{pmatrix}
        \frac{1}{\sqrt{4 + q_R \left(q_R + \sqrt{q_R^2+4}\right)}} \\
        -\frac{1}{\sqrt{2}} \\
        \frac{1}{\sqrt{4 + q_R \left(q_R - \sqrt{q_R^2+4}\right)}}
    \end{pmatrix}
\end{equation*}
for $\Delta(t=0) \ll \Omega$.
In the adiabatic limit, the dressed state populations remain constant and the above superposition evolves under phase acquisition $e^{-i \omega_i t}$ by each dressed state $\ket{i}$~\cite{messiah_quantum_1962}. 
We quantify the stasis of the dressed superposition using the generalized adiabatic parameter $\Gamma \equiv \abs{\vect{\Omega}(t)/\dot{\theta}(t)}$, where $\vect{\Omega} = \vect{B}_{\text{eff}} / \gamma \equiv \Omega \, \vect{e}_x + \Delta \, \vect{e}_x$ and $\tan \theta \equiv \Omega / \Delta$.
For constant coupling amplitude $\dot{\Omega} = 0$, $\dot{\theta}(t) = -\Omega \dot{\Delta}(t)/\abs{\vect{\Omega}}^2$.
For the magnetic field sweeps used here, $\Gamma > 200$ and $\sqrt{\expect{\Gamma^2}}_t \gtrsim 600$ where $\expect{\cdot}_t$ denotes the time-average over the duration of the sweep.
Nevertheless, the continuous spectra when plotted parametrically (\reffig{fig:acquisition_pipeline}(bottom)) exibit some evidence of non-adiabatic following.
This is corroborated by numerically integrating the Schr\"{o}dinger equation of the known control Hamiltonian; the dressed state populations vary slowly (compared to $\omega_D^{-1}$) but by a few percent near minima of $\Gamma(t)$, i.e. when the sweep is least adiabatic.

\section{apparatus}
\label{sec:apparatus}
Our spinor quantum gas apparatus~\cite{wood_magnetic_2015} and Faraday atom-light interface are described in greater detail elsewhere~\cite{jasperse_magic-wavelength_2017}.
We prepare an ultracold gas ($\sim \unit[1]{\upmu K}$) of approximately $10^6$ \Rb atoms in a crossed-beam optical dipole trap ($\lambda=\unit[1064]{nm}$).
The three Zeeman states $\ket{m=-1,0,+1}$ of the lowest hyperfine ($F=1$) ground state are coupled using a radiofrequency field with $\Omega/(2\pi) \leq \unit[100]{kHz}$, generated by a single-turn coil placed immediately atop the glass vacuum cell, fed by an amplified radiofrequency source generated using direct-digital synthesis.
A component of the spin (e.g. $\expect{\hat{F}_x}$) transverse to the static magnetic field direction (along $z$) rotates the polarization of an off-resonant probe beam via the paramagnetic Faraday effect.
    By tuning the probe to a magic-zero wavelength at $\lambda = \unit[790.0]{nm}$, and ensuring it is linearly polarized, the probe exerts no scalar or vector light shift on the atoms.
The former would enact a dipole force on the cloud, perturbing its total density, whereas the latter would be manifest as a fictitious magnetic field and gradient, dephasing the collective spin~\cite{wood_measurement_2016}.
Here we used a wavelength of $\lambda=\unit[781.15]{nm}$ to increase the SNR with respect to the rf pickup at $f_{\text{rf}}$; the measured $1/e$ decay time of calibration peaks $\tau_s = \unit[23.8(2)]{ms}$ is consistent with scattering in~\cite{jasperse_magic-wavelength_2017} for a $\unit[10.6]{mW}$ probe with $1/e^2$ diameter of $\unit[150]{\upmu m}$ (peak intensity $I_0 = \unit[120.1(4)]{W/cm^2}$).


\begin{figure}
    \centering
    \includegraphics[width=\columnwidth]{figure_4.pdf}
    \caption{
    \label{fig:curvature_vs_qR}
        Curvature of the synthetic clock transition for various quadratic shifts $q_R \in [0.2, 0.5]$.
        The measured curvature (black points) was determined from polynomial fitting to $(\Delta B, f_{12})$ data shown in \reffig{fig:acquisition_pipeline}(B).
        Vertical and horizontal error bars correspond to the standard error of the regression and uncertainty in $q_R$ (via $u(q)$ and $u(\Omega)$ at each field $B_0$), respectively.
        A linear fit (black, dashed) with 1--sigma confidence band (gray, shaded) are shown, whose intercept can be used to impute $\qRmagic\text{ (expt.)} = 0.350(6)$.
        The analytic expression for the curvature (red) is consistent with the data-driven analysis of the curvature, \textit{cf.} $\qRmagic\text{ (theory)} = 0.348$.
        The left [right] vertical axis shows the curvature $\partial^2 f_{12}/\partial B^2$ [$\Omega\, \partial^2\omega_{12}/\partial \Delta^2$] in absolute units of $\unit{kHz/G^2}$ [dimensionless units, with the splitting and detuning normalized to the Rabi frequency].
      \textit{cf.} The normalized curvature is unity when $q_R=0$.
    }
\end{figure}

We detect the Faraday rotation of the probe light using a shot-noise limited balanced polarimeter, with bandwidth up to $\unit[8]{MHz}$, and record the signal using an AlazarTech \textsc{ATS9462} digitizer ($16$-bit, $\unit[180]{MS/s}$)
\footnote{The maximum Larmor frequency and thus static magnetic field we can detect Faraday rotation at is limited by the bandwidth of the detector.}.
Upon applying the radiofrequency (rf) dressing field, $\abs{\expect{\hat{F}_x}} > 0$ and the signature of the coupled spin-1 system is a Faraday signal frequency modulated (FM) about a carrier at the Larmor frequency.
The frequency difference of each FM sideband from the carrier is a calibration-free measure of each dressed state splitting $\omega_{ij}$.

As we seek to appraise the robustness of the rf-dressed states to varying magnetic fields, we apply a time dependent $\Delta B(t)$ and observe the dynamical change in the frequency composition of the Faraday signal using the short-time Fourier transform, or spectrogram.
The time-dependent magnetic field shift $\Delta B(t)$ is the sum of an applied linear ramp and the parasitic background fluctuations at the power-line frequency of $\unit[50]{Hz}$ and its odd harmonics, and typically ranges from $0$ ($\Delta = 0$, resonance) to $B_{\text{rf}}$ ($\Delta = 2\Omega$), \textit{cf.} \reffig{fig:eigensystem_schematic}.
For each realization (or `shot') of the experiment, we directly calibrate this time-dependent field using ac magnetometry; an rf $\pi/2$-pulse (rather than continuous coupling) initiates Larmor precession of the collective spin the $x$--$y$ plane, and the Faraday signal is composed of two tones, at $\omega_\pm = \omega_L \pm q$.
For $q \, \tau_f \geq 2\pi$, where $\tau_f$ is the length of the overlapping spectrogram windows, the two tones are spectrally resolved and their mean and difference yields the instantaneous $\omega_L(t)$ and $q(t)$, the former of which is used to find $\Delta B(t)$ by inverting the Breit-Rabi equation~\cite{ramsey_molecular_1956}. 
The experiment is synchronized to the AC power line; the harmonic composition of which varies little between contiguous shots ($\unit[20]{s}$ apart), and thus the measured $\Delta B(t)$ and $q(t)$ from the calibration \note{fiducial?} shot serve as a good proxy for the values experienced by the atoms in the subsequent rf-dressed shot.

We measured the dressed spectrum for a range of $q_R \in [0.2, 0.5]$ by varying the resonant magnetic field $B_0$ [applied rf frequency $f_{\text{rf}}$] from $\unit[3.549]{G}$ [$\unit[2.493]{MHz}$] to $\unit[5.568]{G}$ [$\unit[3.911]{MHz}$], with a fixed Rabi frequency of $\Omega/(2\pi) = \unit[4.505(3)]{kHz}$ ($B_{\text{rf}} = \unit[12.83(1)]{mG}$).
For each resonant field $B_0$, we ensured the Rabi frequency was fixed by measuring the voltage drop across the coil at $f_{\text{rf}}$ with a lock-in amplifier which -- in concert with an impedance analyzer -- could be used to ensure the rf current in the coil and thus $B_{\text{rf}}$ and $\Omega$ were constant. 
The Rabi frequency was ultimately measured using the atoms, by analyzing a subset of the dressed energy spectrum during the magnetic field sweep when $\abs{\Delta}/(2\pi) \leq \unit[100]{Hz}$.
The measured Rabi frequencies had a standard deviation $\sigma(\Omega) = \unit[9.4]{Hz}$, validating the above method.

    Despite long coherence times, the low duty cycle ($D < 0.01$) and large dead time ($T_\text{shot} \gtrsim \unit[10]{s}$) of cold quantum gas experiments make challenging achieving metrological sensitivities per unit bandwidth that are competitive with other platforms.
Here $D=0.005$ and $T_\text{shot} = \unit[20]{s}$, yet we make many more spin projection measurements (\note{$N_m$ = blah} at a shot-noise limited SNR of $10$--$100$~\cite{jasperse_magic-wavelength_2017}) than traditional cold atom experiments ($N_m = 1$ to several, e.g. absorption or dispersive imaging).
This intra-shot revelation of the time and frequency domain renders the measurement of these spectra orders of magnitude more efficient.
For example, the single spectrum shown in \reffig{fig:acquisition_pipeline} would take $\sim (10$ shots per $\Delta B$ per $\omega_{ij} ) \times (100$ different $\Delta B$ values) $\times (3 $ different transition frequencies $\omega_{ij}) = 3000$ shots ($2.5$ times fewer $\Delta B$ values than shown here), or $\sim \unit[6\times10^4]{s} = 1000$ minutes of data acquisition.
We acquire this spectrum in a single shot \note{two shots accounting for the field calibration, but that would require $\sim 2\times$ more traditional shots}, i.e. $\unit[20]{s}$.
The data used to generate \reffig{fig:curvature_vs_qR} was acquired in only $6$ minutes.
  

\section{The data - Think of sexier title}
\label{sec:data}
Our technique directly measures all the quantities in the dressed energy eigen spectrum using sinc fitting to the spectrogram data in a two shot experiment. 
The first observation calibrates essential quantities of the magnetic field ($q_{z}, \omega_{L}, \partialD{B}{t} $) \reffig{fig:acquisition_pipeline}A characterising the horizontal axis of \reffig{fig:eigensystem_schematic}B. In the second shot, we repeat the experiment but dress the atoms with RF \reffig{fig:acquisition_pipeline}B from which we measure the energy splitting. 


\bibliography{dressed_faraday}
   
\end{document}
        
