\documentclass[letterpaper]{article}
\usepackage{upgreek,color}
\usepackage{amsmath,amssymb}
\usepackage{xspace}
\usepackage{units}

%--- MATH OPERATORS ---%
\newcommand{\ket}[1]{\left| #1 \right \rangle \xspace}
\newcommand{\spinup}{\ensuremath{\left| \uparrow \right\rangle \xspace}}
\newcommand{\spindown}{\ensuremath{\left| \downarrow \right\rangle \xspace}}

%--- TEXT SHORTCUTS ---%
\newcommand{\Rb}{$^{87}$Rb\xspace}

% Spacing
\usepackage[left=1.55in,right=1.55in,top=1.2in,bottom=1in]{geometry}
\usepackage{setspace}
\onehalfspacing

\usepackage{titling}
\setlength{\droptitle}{-5em}
\pretitle{\begin{center}\Large}
\posttitle{\par\end{center}}
\postdate{\vspace{-1em}\par\end{center}}

\setlength{\parindent}{0em}
\setlength{\parskip}{1em}

\usepackage{quoting}
\quotingsetup{vskip=-3pt}

\usepackage{enumitem}

% Referee comment environment and response macro
\usepackage{etoolbox}
\usepackage{verbatim}
\makeatletter
\preto{\@verbatim}{\topsep=0pt \partopsep=0pt}
\makeatother

\newenvironment{refcomment}{\singlespacing\verbatim}{\endverbatim}
\newcommand{\response}{\emph{Authors' response---}}

% Notes, etc.
\definecolor{darkgreen}{rgb}{0.01, 0.75, 0.24}
\newcommand{\note}[2][darkgreen]{\textcolor{#1}{[\textrm{#2}]}}
\newcommand{\todo}[1]{{\color{blue}$\blacksquare$~\textsf{[TODO: #1]}}}

% Begin document
\title{Response to Referee Comments: LF16722/Anderson}
\author{R. P. Anderson, M. J. Kewming, and L. D. Turner}

\begin{document}

% \maketitle

\textbf{Response to referee report for manuscript LF16722/Anderson}

December 22, 2017

Dear Dr. Wang,

Thank you for considering our manuscript LF16722 for publication in \textit{Physical Review A} without further review.

Below we respond to the comments made by the referees, and outline associated changes to the manuscript. 

Kind regards,

R. P. Anderson, M. J. Kewming, and L. D. Turner

\newpage
\begin{refcomment}
----------------------------------------------------------------------
Second Report of Referee A -- LF16722/Anderson
----------------------------------------------------------------------

In their reply to my comments the authors boil down my comments to the
fact "that our results lack novelty, because they
essentially replicate the work of Baumgart et al".
First, I want to clarify that I did neither write this in my report
nor did I intend to. Baumgart et al. is just one example out of a host
of papers on dynamical decoupling that I used to stress my point that
this field has been worked on for long. The work of the authors does
add something new, but this in my opinion is just an add-on to
existing work on dynamical decoupling.

In October 2013 Pierre Meystre posted an editorial
(10.1103/PhysRevLett.111.180001), where referees are asked to address
the following points:

"We will ask both authors and referees to address more explicitly than
in the past how the paper (i) substantially advances a particular
field; or (ii) opens a significant new area of research; or (iii)
solves a critical outstanding problem, or makes a significant step
toward solving such a problem; or (iv) is of great general interest,
based, for example, on scientific aesthetics."
\end{refcomment}
\response We apologize for giving the referee the impression that we mis-characterized their first report, and thank them for the helpful explanation of their review in terms of \textit{Physical Review Letters} policies.
[LT: I would just ignore this paragraph and move on]

\begin{refcomment}
Judged the present manuscripts, I get to the following conclusions: i)
Advancing a particular field: Dynamical decoupling is a quite mature
field, and many robust schemes have been developed especially in the
field of NMR and cold ions. To show advancement of another field (e.g.
atomic clocks), the authors would have to prove that their method
allows to surpass the present state-of-the-art. Considering e.g. the
10^-16 sensitivity of state-of-the-art clocks, this is definitely
hard. Considering the example of artificial gauge fields or spinor
dynamics, typical cold atoms experiments currently are limited by
decoherence due to scattering of light, as for state-selective lattice
potentials the detuning can not be larger than the fine-structure
splitting, and not by the residual noise. Actually, the experiments I
know of do not involve dynamical decoupling at all, as this would not
change their results.
\end{refcomment}
\response [Surely ]


\begin{refcomment}
ii) Open a new area: This is clearly not the case, as dynamical
decoupling is quite mature and routinely used in e.g. NMR.

iii) The present manuscripts do not solve an outstanding problem, it
is clearly a subproblem existing in the dynamical decoupling
community. The authors mention "Typical noise levels in
unshielded environments are often comparable to these
Rabi frequency", but unshieded environments can be simply shielded
by mu-metal shieldings, or by using existing pulse sequences (which
admittedly requires control of the pulses, but as e.g. the ion trap
community has shown this is doable to a high precision).
\end{refcomment}
\response
Whilst mu-metal shieldings and pulse sequences can offer robust and simple solutions to noise related issues, in some cases they may not be desirable and a different approach may be required. 

\begin{refcomment}
iv) "is of great general interest": This is of course always a
debatable issue, as "general interest" is a rather unclear definition.
Nevertheless, I doubt that the majority of readers of PRL cares about
the fact that dynamical decoupling now has yet another twist to make
it yet a little better. Thus, I personally would deny the fact that
this is of general interest.
\end{refcomment}
\response
We agree with the reviewer that this is a difficult category to satisfy. Consequently, we acknowledge that our work is of particular interest to those concerned with dynamical decoupling.

\begin{refcomment}
In their reply, the authors state "the transition matrix
elements between dressed states enable the design of
new Hamiltonians with non-trivial topological 
characteristics." This is obviously true, but this is not the focus
of both manuscripts. Thus, if the authors find an interesting
Hamiltonian with non-trivial features I am happy to hear about it, but
the present manuscripts do not provide any insight here. The reply by
the authors did not provide any material that would change my previous
opinion, and correspondingly I still can not recommend publication in
PRL.
\end{refcomment}
\response


\begin{refcomment}
I would also like to mention that for the general audience the
terminology is rather confusing (which is not the authors fault, as
the terms are around already quite some time). The term "continuous
dynamical decoupling" would have been called "dressed state physics" a
decade or two ago, and there is no dynamics required, the pulses are
static. In the end, by adding a control field new eigenstates are
prepared in the system, which have different properties from the bare
states.I fully agree with the authors that the dressed states have
sometimes useful properties (as the quartic decoupling), but this
deviates strongly from the "classical" field of dynamical decoupling,
where a sequence of tailored rephasing pulses was used to battle
decoherence.
\end{refcomment}
\response 
The reviewer has correctly asserted our ``continuous dynamical decoupling'' (CDD) scheme also falls under the category of ``dressed state physics'', but as identified in a later comment and recommended work (Baumgart et al PRL 116, 240801 (2016)) that similar static pulse schemes are regularly referred to as CDD in broader literature as already cited early in our article.
We understand and sympathize with the reviewers terminology concerns which deviate from the classical pulse sequence schemes but we believe it is appropriate to refer to our work as CDD, consistent with the broader literature.


\newpage
\begin{refcomment}
----------------------------------------------------------------------
Report of Referee B -- LF16722/Anderson
----------------------------------------------------------------------

The manuscript describes the generation of dressed states in a spin 1 
quantum gas using continuous dynamical decoupling and shows how 
carefully choosing experimental parameters can lead to
additional protection against magnetic field noise. The manuscript 
also demonstrates a very nice and fast technique for measuring the 
dressed state spectrum as well as the state coherences and coupling 
strengths. This may be particularly interesting for magnetometry. The 
data is well presented, and sufficient error analysis has been 
performed.
\end{refcomment}
\response 


\begin{refcomment}
The use of continuous dynamical decoupling for the protection against 
ambient magnetic field noise is well established and has led to a 
number of publications especially in the fields of NV centres and 
trapped ions and beyond (see references in the manuscript and within).
I particularly note the work of Timoney et al Nature 476, 185 (2011) 
which covers pioneering work in this area, presenting a system which 
is protected against magnetic field fluctuations as well as Rabi 
frequency fluctuations of the dressing fields. I recommend including 
this reference in the manuscript. The work of Baumgart et al PRL 116, 
240801 (2016) is based on the underlying concept of this work and very
nicely makes use of the protected subspace (against magnetic field and
Rabi frequency fluctuations) to demonstrate the suitability of 
continuous dynamical decoupling for magnetometry.
\end{refcomment}
\response 


\begin{refcomment}
The work presented in this manuscript makes a nice addition to the 
broad range of continuous dynamical decoupling work and I believe it 
may find useful applications. However, I do not believe that in its 
current form it is of sufficiently broad interest to warrant 
publication in PRL. While it is of course very interesting to utilize 
‘magic’ experimental parameters to increase the protection against
magnetic field noise and combine this with a very efficient 
measurement technique, given previous work in the area of continuous 
dynamical decoupling which appears to share many similarities with
this work and in some cases has some advantages such as inherent 
protection against dressing field amplitude noise I believe the 
advances in this manuscript are not sufficient and more is needed to
warrant publication in PRL. As stated in the introduction and 
conclusion, one of the prime potential applications is magnetometry. 
To be of broader interest I believe significantly more work is 
required to show the suitability of this work for magnetometry and 
beyond as it is not clear at all from the manuscript how well this 
system will perform in this and other potential application areas.
\end{refcomment}
\response


\begin{refcomment}
I do believe the work in its current form is of interest and in my 
opinion, would fit well with the requirements of NJP or PRA for 
example.
\end{refcomment}
\response 


\end{document}
