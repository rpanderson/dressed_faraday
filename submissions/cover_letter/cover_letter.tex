%%%%%%%%%%%%%%%%%%%%%%%%%%%%%%%%%%%%%%%%%
% Professional Formal Letter
% LaTeX Template
% Version 2.0 (12/2/17)
%
% This template originates from:
% http://www.LaTeXTemplates.com
%
% Authors:
% Brian Moses
% Vel (vel@LaTeXTemplates.com)
%
% License:
% CC BY-NC-SA 3.0 (http://creativecommons.org/licenses/by-nc-sa/3.0/)
%
%%%%%%%%%%%%%%%%%%%%%%%%%%%%%%%%%%%%%%%%%

%----------------------------------------------------------------------------------------
%	PACKAGES AND OTHER DOCUMENT CONFIGURATIONS
%----------------------------------------------------------------------------------------

\documentclass[10pt,letterpaper]{letter} % Set the font size (10pt, 11pt and 12pt) and paper size (letterpaper, a4paper, etc)

\input{structure.tex} % Include the file that specifies the document structure

\longindentation=0pt % Un-commenting this line will push the closing "Sincerely," and date to the left of the page

%----------------------------------------------------------------------------------------
%	YOUR INFORMATION
%----------------------------------------------------------------------------------------

\Who{Dr Russell Anderson} % Your name

\Title{} % Your title, leave blank for no title

\authordetails{
	% School of Physics and Astronomy\\ % Your department/institution
	19 Rainforest Walk\\ % Your address
	Monash University \\
	Victoria, Australia 3800\\ % Your city, zip code, country, etc
	% russell.anderson@monash.edu\\ % Your email address
	% (+61) 3-9905-5943 \\ % Your phone number
	% URL: physics.monash.edu % Your URL
}

%----------------------------------------------------------------------------------------
%	HEADER CONTENTS
%----------------------------------------------------------------------------------------

\logo{monash_logo.pdf} % Logo filename, your logo should have square dimensions (i.e. roughly the same width and height), if it does not, you will need to adjust spacing within the HEADER STRUCTURE block in structure.tex (read the comments carefully!)

\headerlineone{} % Top header line, leave blank if you only want the bottom line

\headerlinetwo{} % Bottom header line

%----------------------------------------------------------------------------------------

\begin{document}

%----------------------------------------------------------------------------------------
%	TO ADDRESS
%----------------------------------------------------------------------------------------

\begin{letter}{
	Physical Review Letters\\
	Editorial Office\\
	1 Research Road \\
	Ridge, NY 11961-2701\\
}

%----------------------------------------------------------------------------------------
%	LETTER CONTENT
%----------------------------------------------------------------------------------------

\opening{Dear Editors,}

We are pleased to submit our manuscript 'Continuously observing a dynamically decoupled spin-1 quantum gas' to \textit{Physical Review Letters}.
In it we observe in real-time the continuous dynamical decoupling of a spin-1 quantum gas, probing its spectrum, coherences, and coupling strengths.
Dynamical decoupling a system from its environment is not only applicable to quantum information but has broad application to quantum metrology and quantum state estimation.
Yet few systems permit dynamical decoupling to be observed in real-time; weak measurements of superconducting qubits and nitrogen-vacancy (NV) centers require thousands of repeated preparation and detection cycles to measure what we do in a single-preparation of the quantum gas.

This work weds clock-state metrology with continuous measurement, ingredients for new experiments that emulate quantum magnetism in solid-state systems, and advancing a platform for measuring oscillating magnetic fields in noisy environments.
More broadly still, the decoupled microscale quantum gas offers magnetic sensitivity in a tunable kilohertz band, persistent over many milliseconds: this is the length scale, frequency, and duration relevant to sensing biomagnetic phenomena such as neural spike trains.

We advance spectrogram analysis as a time-frequency reduction of the measurement record to extract rich information about the quantum state and Hamiltonian simultaneously.
Spectrograms have been applied widely, from visualizing gravitational wave signals to analyzing human speech.
We `listen' to the precessing spins of the decoupled quantum gas, and use spectrogram analysis to parse the recorded tones as a signature of dynamical decoupling.

Our manuscript forms a co-submission with D. Trypogeorgos \textit{et al.}, who present a complementary characterization of the high-order decoupling manifest in this system as insensitivity of the states to magnetic field variations.
A distinguishing focus of our work is the continuous weak measurement, which could give rise to new forms of quantum sensing exploting synchronous detection and feedback.
Additionally, we demonstrate the ability to probe dressed-state coupling strengths without driving transitions, a form of Hamiltonian parameter estimation.
Our work is further distinguished by new analytic results identifying where the optimum decoupling occurs, applicable to any multi-level system.

\closing{Sincerely,}

%----------------------------------------------------------------------------------------
%	OPTIONAL FOOTNOTE
%----------------------------------------------------------------------------------------

% Uncomment the 4 lines below to print a footnote with custom text
%\def\thefootnote{}
%\def\footnoterule{\hrule}
%\footnotetext{\hspace*{\fill}{\footnotesize\em Footnote text}}
%\def\thefootnote{\arabic{footnote}}

%----------------------------------------------------------------------------------------

\end{letter}

\end{document}
