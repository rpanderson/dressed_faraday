%--- PACKAGES ---%
% \usepackage[T1]{fontenc}
\usepackage{graphicx}
\usepackage[usenames,dvipsnames]{color}
\usepackage{amsmath,amssymb}
\usepackage{bm}
\usepackage{upgreek}
\usepackage{xspace}
\usepackage{units}
\usepackage{hhline}
\usepackage[vskip=0pt]{quoting}
\usepackage[colorlinks,urlcolor=blue,citecolor=blue,linkcolor=blue]{hyperref}
\graphicspath{{figures/}}
\renewcommand{\arraystretch}{1.5}

%--- TEXT SHORTCUTS ---%
\newcommand{\etal}{et~al.\xspace}
\newcommand{\Rb}{$^{87}$Rb\xspace}
\newcommand{\qmagic}{q_{\text{magic}}\xspace}
\newcommand{\qRmagic}{q_{R,\text{magic}}\xspace}
% \newcommand{\lundblad}{Phys. Rev. Lett. \textbf{118}, 2xxxxx (2017)}
\newcommand{\lundblad}{arXiv:1706.xxxx}

%--- TEXT OPERATORS ---%
\newcommand{\reffig}[1]{\mbox{Fig.~\ref{#1}}}
\newcommand{\refeq}[1]{\mbox{Eq.~(\ref{#1})}}
\newcommand{\refsec}[1]{\mbox{Sec.~(\ref{#1})}}
\newcommand{\note}[1]{\textcolor{ForestGreen}{[\textrm{#1}]}} % Make editorial notes red

%--- MATH OPERATORS ---%
\newcommand{\upd}{\text{d}}
\newcommand{\totalD}[2]{\frac{\upd #1}{\upd #2}}
\newcommand{\partialD}[2]{\frac{\partial #1}{\partial #2}}
\newcommand{\vecop}[1]{\hat{\mathbf{#1}}\xspace}
\newcommand{\expect}[1]{\langle #1 \rangle}
\newcommand{\abs}[1]{\vert #1 \vert \xspace}
\newcommand{\bra}[1]{\langle #1 \vert \xspace}
\newcommand{\ket}[1]{\vert #1 \rangle \xspace}
\newcommand{\braket}[2]{\langle #1 \vert #2 \rangle \xspace}
\newcommand{\vect}[1]{\mathbf{#1}\xspace}
\newcommand{\uvect}[1]{\hat{\mathbf{#1}}\xspace}
\newcommand{\epvec}{\hat{\mathbf{\epsilon}}\xspace}
\newcommand{\jvect}{\mathbf{\tilde{E}}\xspace}
\newcommand{\ham}{\mathcal{H}}

% Underscores in text (from http://tex.stackexchange.com/a/38720)
\catcode`_=12
\begingroup\lccode`~=`_\lowercase{\endgroup\let~\sb}