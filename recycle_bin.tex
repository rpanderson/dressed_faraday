\textit{Analytic Faraday signal on resonance:}
\begin{table}[t]
    \caption{Upper sidebands of the carrier (at $\omega_{\text{rf}}$) of the Faraday rotation signal $\propto \expect{\hat{F}_x}$ of a state $\ket{\psi(t=0)}=\ket{m=1}$ driven on resonance ($\Delta = 0$) in the laboratory frame. Frequency and phase are reported relative to the carrier, along with the transition that each sideband corresponds to. For each upper sideband, there is a lower sideband of the same amplitude, relative frequency and opposite \note{$\pi$} relative phase.}
    \label{tab:sidebands}
    \begin{tabular}{ccc}
    \hhline{===}
    frequency & amplitude & transition \\ \hhline{---}
     $0$ & $\frac{q_D \Omega }{2 \omega_D^2}$ & -- \\
     $\omega_D-q_D$ & $\frac{\Omega}{4 \omega_D}$ & $\ket{1} \leftrightarrow \ket{2}$ \\
     $\omega_D+q_D$ & $\frac{\Omega}{4 \omega_D}$ & $\ket{2} \leftrightarrow \ket{3}$\\
     $2 \omega_D$ & $\frac{q_D \Omega}{4 \omega_D^2}$ & $\ket{1} \leftrightarrow \ket{3}$ \\ \hhline{===}
    \end{tabular}
\end{table}


The dressed-state transitions are thus cyclic and non-degenerate (\reffig{fig:eigensystem_schematic}(right)), characterized by a dressed Larmor frequency
\begin{align}
\label{eq:dressed_larmor}
   \omega_D \equiv (\omega_3 - \omega_2)_{\Delta=0}/2 &= (\omega_{12} + \omega_{23})_{\Delta=0}/2 \notag\\ &= \sqrt{\Omega^2 + q_D^2}\, ,
\end{align}
and dressed quadratic shift
\begin{align}
\label{eq:dressed_quadratic}
   q_D &\equiv (\omega_3 + \omega_1 -2\omega_2)_{\Delta=0}/2 \notag\\
       &= (\omega_{23}-\omega_{12})_{\Delta=0}/2 \notag s\\ 
       &= -q/2 \, .
\end{align}

We characterize the splittings between dressed states at $\Delta=0$ by a dressed Larmor frequency $\omega_D\equiv(\omega_3-\omega_1)/2=\sqrt{\Omega^2+q_D^2}$, and dressed quadratic shift $q_D \equiv (\omega_3 + \omega_1 -2\omega_2)/2=-q/2$, giving splittings $\omega_{23}=\omega_D-q_D$, $\omega_{12}=\omega_D+q_D$ and $\omega_{13}=2\omega_D$.


% ADIABATIC CONSIDERATIONS (FINAL DRAFT)
In the adiabatic limit, the dressed state populations $\rho_{ii}$ remain constant and each dressed state $\ket{i}$ acquires phase $e^{-i \omega_i t}$~\cite{messiah_quantum_1962}. 
We quantify the stasis of the dressed superposition using the adiabatic parameter $\Gamma \equiv \abs{\vect{\Omega}(t)/\dot{\theta}(t)}$, where $\tan \theta \equiv \Omega / \Delta$.
(The adiabatic limit is $\Gamma\gg 1$.)
For constant coupling amplitude $\dot{\Omega} = 0$, $\dot{\theta}(t) = -\Omega \dot{\Delta}(t)/\abs{\vect{\Omega}}^2$.
For the magnetic field sweeps used here, $\Gamma$ exceeded 200 at all times and was on average $\gtrsim 600$.
Nevertheless, the continuous spectra when plotted parametrically (\reffig{fig:acquisition_pipeline}c) exhibit some evidence of non-adiabatic following.
Numerical integration of the Schr\"{o}dinger equation of the known control Hamiltonian confirms that dressed state populations vary slowly (compared to $\omega_D^{-1}$) and deviate by no more than several percent from their initial values near minima of $\Gamma(t)$.